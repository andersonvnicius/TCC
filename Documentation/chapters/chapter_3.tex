
\phantomsection

\chapter{Resultados}\label{ch:capitulo_resultados}

Este capítulo apresenta os sinais e valores obtidos no experimento realizado no dispositivo de flexão.
Os tópicos aqui analisados apresentam a comparação dos resultados obtidos pelo experimento realizado pelo autor com os valores obtidos pelo
estudo de caso feito por Minela (2017).

O objetivo primário da comparação dos resultados é o de se obter dados descritivos de desempenho do dispositivo desenvolvido em relação a um sistema de medição industrial
homologado, no final do capítulo são indicados as situações de melhor desempenho do protótipo.

\section{Sinais obtidos}

Os sinais captados pelo sistema de medição desenvolvido seguem um formato trapezoidal, onde as zonas iniciais e finais representam os momentos em que a viga não se encontrava
sob a aplicação da carga, e a zona intermediária representa a total aplicação da carga no dispositivo.

\subsection{Sinais de calibração}

A primeira etapa da utilização do dispositivo é definir os sinais de calibração, para isso é obtido as respostas de leitura do módulo de conversão de sinal para a aplicação das cargas
de 0,138kg e 1,04kg, que representavam o menor e o maior peso disponível para o experimento, sem considerar as combinações.

Foi obtido o valor médio de 190 unidades discretas na zona superior do sinal trapezoidal na aplicação da massa de 0,138kg, como mostrado na \autoref{fig:4100}.
Para a aplicação da massa de 1.04kg foi obtido o valor de 612 unidades discretas, mostrado na \autoref{fig:420}.

\begin{figure}[H]
	\caption{\label{fig:4100} Sinal obtido pela aplicação da carga de 0,138kg no dispositivo}
	\begin{center}
		\includegraphics[width=350]{pictures/signals/cal_signal_1.png}
	\end{center}
	\fonte{O autor 2022}
\end{figure}

\begin{figure}[H]
	\caption{\label{fig:420} Sinal obtido pela aplicação da carga de 1,04kg no dispositivo}
	\begin{center}
		\includegraphics[width=350]{pictures/signals/cal_signal_2.png}
	\end{center}
	\fonte{O autor 2022}
\end{figure}

Uma vez obtidos os valores discretos para cada ponto de calibração e conhecidos os seus valores de deformação teóricos, é calculado os fatores $ fa $ e $ fb $ da função de interpolação
linear que converte valores obtidos pelo módulo de conversão de sinal em unidades discretas para valores equivalentes de deformação no extensômetro.
O valor do fator a encontrado foi de $ 2.8881*10^{-6} $ e o de b foi de $ -363.4311*10^{-6} $.

Uma vez definidos os parâmetros de calibração são feitos os experimentos aplicando a carga de cada peso no dispositivo de flexão.
Alguns dos sinais obtidos e uma discussão geral sobre eles são discutidos nas seções a seguir.

\subsection{Sinais calibrados obtidos}

O experimento é realizado obtendo os sinais calibrados obtidos pelo dispositivo na aplicação de cada uma das massas disponíveis e pela aplicação de combinações entre essas massas,
os valores de cargas aplicadas para a realização dos experimentos são mostradas na \autoref{tab:Cargasaplicadas}.

\begin{table}[H]
    \caption{Cargas aplicadas para os experimentos}
    \label{tab:Cargasaplicadas}
    \centering
    \resizebox{350}{!}{%
        \begin{tabular}{ l c r } \toprule
			Pesos aplicados & {Massa aplicada (Considerando fuso e porca)} \\
			Peso 1 & ${138.42} {g} $ \\
			Peso 2 & ${250.07} {g} $ \\
			Peso 3 & ${1048.82} {g} $ \\
			Peso a & ${549.35} {g} $ \\
			Pesos 1 e 2 & ${336.8} {g} $ \\
			Pesos 1 e a & ${636.08} {g} $ \\
			Pesos 2 e a & ${747.73} {g} $ \\
			Pesos 1, 2 e a & ${834.46} {g} $ \\
			Pesos 3 e 1 & ${1135.55} {g} $ \\
			Pesos 3 e 2 & ${1247.2} {g} $ \\
			Pesos 3, 2 e 1 & ${1333.93} {g} $ \\
            \bottomrule
        \end{tabular}}
\fonte{O autor 2022}
\end{table}

O primeiro peso aplicado é o de menor massa disponível, que ja tinha sido previamente utilizada para utilização na calibração do dispositivo,
o sinal obtido é mostrado na \autoref{fig:41002}, o valor de deformação obtido foi de $ 182.42 {\mu}m/m $.

\begin{figure}[H]
	\caption{\label{fig:41002} Sinal obtido da aplicação de 0,138 kg de carga aplicada no dispositivo}
	\begin{center}
		\includegraphics[width=350]{pictures/signals/w1.png}
	\end{center}
	\fonte{O autor 2022}
\end{figure}

O sinal equivalente obtido por Minela (2017) é mostrado na \autoref{fig:4102_m}.

\begin{figure}[H]
	\caption{\label{fig:4102_m} Sinal obtido da aplicação da massa de 0,163 kg de carga aplicada no dispositivo indústrial}
	\begin{center}
		\includegraphics[width=350]{pictures/signals/wp1.png}
	\end{center}
	\fonte{\autocite{Minela2017}}
\end{figure}

Uma comparação direta de valor não pode ser feita considerando os dois sinais, uma vez que as massas aplicadas não foram as mesmas, porém, analisando os sinais
pode-se notar que o efeito dos ruídos no dispositivo desenvolvido é consideravelmente menor do que os obtidos pelo dispositivo utilizado em Minela (2017).
Foi constatado que os dados apresentados por Minela (2017) foram obtidos sem a ativação dos métodos de redução de ruídos disponíveis no dispositivo Lynx ADS2002,
logo, pode-se concluir que a atuação dos filtros de passa-altas do módulo de conversão de sinal utilizado no dispositivo desenvolvido é o principal motivo pelos quais
os ruídos em todos os resultados obtidos pela utilização do protótipo desenvolvido se mostraram de baixas amplitudes.

O sinal obtido pela aplicação do de 1,04kg é mostrado na \autoref{fig:4109}, o valor de deformação obtido foi de $ 1421.41 {\mu}m/m $.

\begin{figure}[H]
	\caption{\label{fig:4109} Sinal obtido da aplicação de 1,040 kg de carga aplicada no dispositivo}
	\begin{center}
		\includegraphics[width=350]{pictures/signals/w8.png}
	\end{center}
	\fonte{O autor 2022}
\end{figure}

O sinal equivalente obtido por Minela (2017) é mostrado na \autoref{fig:4109_m}

\begin{figure}[H]
	\caption{\label{fig:4109_m} Sinal obtido da aplicação da massa de 0,163 kg de carga aplicada no dispositivo indústrial}
	\begin{center}
		\includegraphics[width=350]{pictures/signals/wp3.png}
	\end{center}
	\fonte{\autocite{Minela2017}}
\end{figure}

Assim como no primeiro sinal, nota-se que os sinais são semelhantes, com a exceção dos ruídos, que são mais impactantes com 
a utilização do dispositivo utilizado por Minela (2017) do que com a utilização do protótipo desenvolvido.

Os sinais obtidos para cada carga utilizada no experimento podem ser encontrados no \autoref{ch:sinais-obtidos}.

\subsection{Comparação dos valores de deformação experimentais e analíticos}

Utilizando a \autoref{eq:Eq_201} pode-se obter analiticamente valores de deformação esperados para cada situação de carga do experimento.
O valor de módulo de elasticidade utilizado no modelo analítico foi de $ 85.19 GPa $, que foi obtido experimentalmente no trabalho de Minela (2017)
para o mesmo dispositivo utilizado.
A \autoref{tab:ResultadosDeformacao} mostra os resultados de valores de deformação obtidos para cada experimento, os valores de deformação esperados
calculados e os valores de erro de cada resultado obtido em relação ao teórico.

\begin{table}[H]
    \caption{Resultados de deformação obtidos}
    \label{tab:ResultadosDeformacao}
    \centering
    \resizebox{350}{!} $ \\
			$ {250.07} {g} $ & $ {334.77} {\mu}m/m $ & $ {358.59} {\mu}m/m $ & $ {7.12\%} $ \\
			$ {336.8} {g} $ & $ {450.88} {\mu}m/m $ & $ {474.12} {\mu}m/m $ & $ {5.15\%} $ \\
			$ {549.35} {g} $ & $ {735.43} {\mu}m/m $ & $ {760.04} {\mu}m/m $ & $ {3.35\%} $ \\
			$ {636.08} {g} $ & $ {851.53} {\mu}m/m $ & $ {884.22} {\mu}m/m $ & $ {3.84\%} $ \\
			$ {747.73} {g} $ & $ {1001.00} {\mu}m/m $ & $ {1017.08} {\mu}m/m $ & $ {1.61\%} $ \\
			$ {834.46} {g} $ & $ {1117.11} {\mu}m/m $ & $ {1144.15} {\mu}m/m $ & $ {2.42\%} $ \\
			$ {1048.82} {g} $ & $ {1404.08} {\mu}m/m $ & $ {1421.41} {\mu}m/m $ & $ {1.23\%} $ \\
			$ {1135.55} {g} $ & $ {1520.19} {\mu}m/m $ & $ {1542.71} {\mu}m/m $ & $ {1.48\%} $ \\
			$ {1247.2} {g} $ & $ {1669.66} {\mu}m/m $ & $ {1701.55} {\mu}m/m $ & $ {1.91\%} $ \\
			$ {1333.93} {g} $ & $ {1785.76} {\mu}m/m $ & $ {1811.3} {\mu}m/m $ & $ {1.43\%} $ \\
			\bottomrule
        \end{tabular}}
\fonte{O autor 2022}
\end{table}

Analisando os resultados pode-se notar que os valores de erro são mais altos nas situações de menores cargas aplicadas, e diminuem conforme a carga aplicada aumenta.

\subsection{Comparação dos valores de deformação obtidos pelo dispositivo e por Minela (2017)}

Para poder se fazer uma comparação direta entre os resultados obtidos pelo autor com os do trabalho de Minela (2017), foi aplicado fatores de correção nos resultados de deformação
obtidos, que equivalem a compensação da variação da massa dos componentes previamente apresentados na \autoref{tab:MassasUtilizadas}.
A \autoref{tab:ResultadosComparacao} compara os principais resultados corrigidos obtidos pelo experimento desenvolvido pelo autor com os resultados obtidos pelo trabalho desenvolvido por Minela (2017).

\begin{table}[H]
    \caption{Comparação entre os resultados obtidos com os obtidos por Minela (2017)}
    \label{tab:ResultadosComparacao}
    \centering
    \resizebox{\linewidth}{!} $ \\
			$ {261.62} {g} $ & $ {373.72} {\mu}m/m $ & $ {357.53} {\mu}m/m $ & $ {4.33} {\%} $ \\
			$ {1060.56} {g} $ & $ {1520.59} {\mu}m/m $ & $ {1433.79} {\mu}m/m $ & $ {5.71} {\%} $ \\
			$ {1160.24} {g} $ & $ {1574.89} {\mu}m/m $ & $ {1567.61} {\mu}m/m $ & $ {0.46} {\%} $ \\
			$ {1258.92} {g} $ & $ {1716.18} {\mu}m/m $ & $ {1720.84} {\mu}m/m $ & $ {0.27} {\%} $ \\
			$ {1358.6} {g} $ & $ {1843.45} {\mu}m/m $ & $ {1818.79} {\mu}m/m $ & $ {1.34} {\%} $ \\
            \bottomrule
        \end{tabular}}
\fonte{O autor 2022}
\end{table}

Nota-se que assim como a comparação com os resultados analíticos, também foi observado as altas variações na aplicação de baixas cargas e erros que se diminuem com as maiores cargas
aplicadas.
A \autoref{fig:430} faz a comparação final entre os resultados obtidos pelo protótipo do dispotivo desenvolvido e pelo dispositivo utilizado por Minela (2017).

\begin{figure}[H]
	\caption{\label{fig:430} Comparação final dos resultados obtidos}
	\begin{center}
		\includegraphics[width=\linewidth]{pictures/final_chart.png}
	\end{center}
	\fonte{O autor 2022}
\end{figure}

%%TODO falar que os resultados não foram obtidos de maneira sem fio e que não se pode garantir que de modo sem fio vão ser livres de efeitos de ruidos de transmissão
