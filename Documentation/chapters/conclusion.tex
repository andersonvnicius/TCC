++
\phantomsection

\chapter{Considerações Finais}

O objetivo geral do trabalho foi considerado como alcançado, uma vez que foi desenvolvido com sucecsso um protótipo funcional.
Por mais que não foi possível a sua aplicação em um eixo rotacional, o seu conceito de funcionamento foi validado por um experimento estático.

O objetivos específico de desenvolver um protótipo focado na utilização por um público alvo foi parcialmente alcançado.
As etapas e as decisões das fases da metodologia de projeto foram tomadas considerando os requisitos dos clientes, porém não foi possível validar
um dos componentes essenciais do funcionamento do dispositivo na etapa de experimentação, os testes foram realizados utilizando um módulo de conversão
de sinal diferentes do definido no projeto conceitual.

As documentações necessárias para replicar o dispositivo se encontram nos \autoref{ch:projeto-elétrico}, \autoref{ch:algoritmo-de-programação-do-controlador-do-dispositivo}
e \autoref{ch:algoritmos-desenvolvidos-para-o-software-de-conexão}.
Porém o projeto detalhado, principalmente o projeto mecânico ainda pode ser consideravelmente melhorado adicionando com maior detalhe os encaixes para os componentes de alimentação
e os encaixes dos módulos no encapsulamento do dispositivo.

Os valores obtidos pelo experimento se mostraram com erros aceitáveis para aplicações de baixo custo, principalmente quando comparados com os
valores obtidos por um dispositivo homologado, logo, a precisão do dispositivo foi considerada como aceitável para a utilização por equipes de competição,
uma vez que atualmente é observado a dificuldade pela obtenção desses tipos de dados.

Os dados obtidos pelo dispositivo ainda podem apresentar melhores valores de faixa de leitura e precisão, para isso é sugerido a implementação de um método de zerar a tensão
inicial da ponte de Wheatstone substituindo um dos resitores da ponte por elemento de resistência variável que pode ser ajustado pelo usuário, como um resitor trimpot.

Também deve-se apontar que uma equipe de competição provavelmente já estará utilizando dispositivos para obtenção de dados de sensor, logo, pode-se integrar o dispositivo desenvolvido
á um dispositivo central de obtenção, caso esse suporte uma obtenção de dados utilizando comunicação via rede sem fio.

O software de obtenção de dados desenvolvido pode ser facilmente modificado e expandido para a representação de diferentes tipos de dados, logo, ele pode ser aplicado para aplicações
além de obter os dados do dispositivo desenvolvido.

O Método de obtenção de dados via rede de internet faz com que o dispositivo possa ser utilizado para aplicações remotas, como em aplicações de telemetria industrial, porém
deve-se notar as limitações em relação a precisão dos dados obtidos.

O preço final encontrado do dispositivo se mostra consideravelmente menor do que os dos dispositivos encontrados no mercado, porém, deve-se notar que os módulos utilizados são
desenvolvidos para aplicações específicas e normalmente utilizados por entusiastas em pequenos projetos, logo, não se pode confirmar a confiabilidade e precisão ao longo do tempo,
assim como a durabilidade de um dispositivo fabricado.

\subsection{Trabalhos futuros sugeridos}

Possiveis melhorias do dispositivo e expansão do estudo desenvolvido pelo autor são apresentados na listagem a seguir:

\begin{alineas}
    \item{Utilizar o dispositivo para obter os dados em uma aplicação veícular}
    \item{Estudar o efeito do desbalanceamento de massa do dispositivo no eixo em que está acoplado}
    \item{Estudar o impacto da utilização de diferentes elementos extensômetros na obtenção do sinal pelo dispositivo}
    \item{Realizar o experimento utilizando uma ponte de Wheatstone composta unicamente por extensômetros}
    \item{Realizar o experimento utilizando o dispositivo de torção desenvolvido por Minela (2017)}
    \item{Obter os valores de deformação de um eixo utilizando primariamente os valores de tensão obtidos utilizando uma função de transferência}
    \item{Estudar efeitos de atraso na transfmissão dos sinais por comunicação sem fio}
    \item{Desenvolver método para transmissão de dados em longas distâncias e em locais que não possuem redes de internet sem fio disponíveis}
\end{alineas}
