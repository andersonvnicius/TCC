
\phantomsection

\chapter{\lang{Introduction}{Introdução}}
\phantomsection

Atualmente, pode-se observar o constante aumento da importância da otimização dos projetos de componentes em projetos de produtos na indústria automotiva, produtos altamente otimizados
resultam em um menor custo de material e de fabricação dos componentes.

A carga presente em um componente em operação pode ser diferente que as cargas previstas no seu projeto, isso pode ocorrer devido a erros durante a fabricação ou na montagem dos componentes.
Para se garantir a segurança, um componente devem ser projetado para suportar tensões admissíveis maiores do que as presentes durante a sua utilização. \autocite{Hibbeler2010}

Como resposta às essas incertezas envolvidas no projeto analítico de um componente os engenheiros tendem a selecionar maiores valores de fator de segurança em um projeto, o que causa o aumento
de custo e de massa de um componente.
Um método de evitar os altos valores de fatores de segurança é a utilização de dados de cargas que representam com mais exatidão as situações reais de cargas aplicadas em um componente.

Dados reais de utilização podem ser obtidos utilizando dispositivos de sensoriamento, porém, nota-se que ainda existe uma dificuldade de obtenção de alguns tipos de dados de maneira direta
em um veículo, como os dados de cargas de tração em um eixo em movimento \autocite{Nurprasetio2018}.
Um dos motivos desse problema é devido ao fato de que os equipamentos disponíveis para se medir são usualmente para aplicações de alta precisão como os transdutores de torque,
o princípio de funcionamento de um transdutor de torque é mostrado na \autoref{fig:0100}.

\begin{figure}[htb]
	\caption{\label{fig:0100} Princípio de funcionamento de um transdutor de torque}
	\begin{center}
		\includegraphics[width=\textwidth]{pictures/0100.png}
	\end{center}
	\fonte{adaptado de \autocite{Kyowa}}
\end{figure}


O elemento sensor em um transdutor de torque são sensores de deformação montados no eixo, esse tipo de sensor é utilizado uma vez que as deformações locais em um corpo são proporcionais às
cargas presentes no corpo caso o material esteja operando em condição de deformação elástica.
Extensômetros, ilustrado na \autoref{fig:0200}, são sensores que, quando colados á uma superfície apresentam variação de suas resistências elétricas caso ocorra a deformação local na
superfície que estám colados.
Esses tipos de sensores apresentam uma boa disponibilidade no mercado e são amplamente utilizados em balanças e células de carga.

\begin{figure}[H]
	\caption{\label{fig:0200} Extensômetro}
	\begin{center}
		\includegraphics[width=290]{pictures/0200.png}
	\end{center}
	\fonte{\autocite{imsg}}
\end{figure}

O presente trabalho propõe o desenvolvimento do projeto de um dispositivo de baixo custo para obtenção de dados de torque em um eixo em movimento, de forma que os sinais obtidos pelo dispositivo
sejam transmitidos utilizando tecnologias de comunicação sem fio, para eliminar a necessidade dos conectores elétricos rotativos.
O desenvolvimento do projeto do dispositivo seguirá a metodologia de projeto de produto PRODIP com o objetivo de garantir replicabilidade, permitir futuras otimizações e expansões e facilitar sua
implementação em um caso real.

Por fim, será montado o protótipo do dispositivo com a finalidade de validar seu princípio de funcionamento e verificar a precisão dos dados obtidos com sua utilização, que é feito seguindo a
metodologia de experimentos e com a comparação dos resultados de dados de deformação obtidos por um dispositivo industrial nos estudos de caso desenvolvidos por Minela (2017).

\section{Objetivos}

Os objetivos do trabalho são apresentados nas seções a seguir.

\subsection{Objetivo Geral}

Desenvolver um dispositivo de baixo custo para obtenção de dados de torque em componentes rotativos.

\subsection{Objetivos Específicos}

$\bullet$ Utilizar ferramentas de pesquisa informacional para definir um público alvo e seus requisitos de produto.

$\bullet$ Realizar uma fase de desenvolvimento conceitual para selecionar os componentes.

$\bullet$ Desenvolver as documentações necessárias para garantir a replicabilidade e expansibilidade do dispositivo.

$\bullet$ Executar teste experimental para obter dados de precisão do dispositivo.
