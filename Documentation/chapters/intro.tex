
\phantomsection

\chapter{\lang{Introduction}{Introdução}}
\phantomsection

Atualmente, é notado o constante aumento da importância da otimização dos projetos de componentes em projetos de produtos na indústria automotiva, produtos altamente otimizados resultam em um menor custo de material e de fabricação dos componentes, \autocite{Hibbeler2010} afirma que “a carga para a qual um elemento é projetado pode ser diferente das cargas realmente aplicadas. As dimensões estipuladas no projeto de uma estrutura ou máquina podem não ser exatas, na realidade, por causa de erros de fabricação ou cometidos na montagem de seus componentes”.

Segundo \autocite{Hibbeler2010} “Para se garantir a segurança, é preciso escolher uma tensão admissível que restrinja a carga aplicada a um valor menor do que a carga que o elemento pode suportar totalmente.” Então como resposta às incertezas envolvidas no projeto analítico de um componente os projetistas devem projetar componentes que suportam forças superiores às presentes na utilização do componente. O que resulta em altos valores de fator de segurança em um projeto, o que causa impacto monetário e aumento de massa do componente. Uma das maneiras que permite a diminuição de valores de fator de segurança é a alimentação do projeto do componente com dados de cargas que representam o mais próximo o possível aos presentes na situação real.

Dados reais de utilização podem ser obtidos por sensores em componentes reais ou de teste submetidos a situações reais, porém atualmente certos parâmetros não podem ser facilmente medidos de maneira direta em um veículo, dentre eles forças normais e forças torcionais \autocite{Nurprasetio2018}. Hibbeler afirma que “as medições de deformação são experimentais e, uma vez obtidas, podem ser relacionadas com as cargas aplicadas, ou tensões, que agem no interior do corpo.” Logo conclui-se que uma maneira direta de medir as forças internas atuantes em um componentes é obtendo os dados de deformação local.

Foi observado que dispositivos utilizados para obter dados de deformação em tempo real com precisão são usualmente utilizados em testes de impacto e de controle de qualidade em componentes pela indústria automotiva, esses dispositivos apresentam altos níveis de precisão e confiabilidade e, consequentemente altos custos, o que inviabiliza sua utilização fora do produto final. Os valores de deformação local em um componente podem ser obtidos utilizando sensores de deformação chamados de extensômetros, esses sensores apresentam uma boa disponibilidade no mercado e são amplamente utilizados em células de carga. Os sinais gerados por esse tipo de sensor devem ser instrumentados, ampliados e convertidos para possibilitar sua obtenção por uma interface controladora.

O presente trabalho propõe o desenvolvimento de um protótipo de um dispositivo de baixo custo para obtenção de dados de deformação em componentes. O desenvolvimento do dispositivo seguirá a metodologia de projeto de produto PRODIP com o objetivo de garantir replicabilidade, permitir futuras otimizações e expansões e facilitar sua implementação em um caso real. Por fim, o funcionamento, efetividade e precisão do protótipo desenvolvido será avaliado comparando dados obtidos pelo protótipo e por um dispositivo industrial homologado, seguindo a metodologia de \autocite{Minela2017}.

\section{Objetivos}

Os objetivos do trabalho são apresentados nas seções a seguir.

\subsection{Objetivo Geral}

Desenvolver um dispositivo de baixo custo para obtenção de dados em tempo real de deflexão em componentes mecânicos.

\subsection{Objetivos Específicos}

Obter dados de deflexão em vigas

Obter módulo de elasticidade de uma liga desconhecida

Desenvolver utilizando tecnologias de código aberto

Obter valores de precisão do protótipo desenvolvido

