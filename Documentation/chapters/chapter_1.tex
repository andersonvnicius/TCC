%% chapters/chapter_1.tex
%%
%% Copyright 2017 Evandro Coan
%% Copyright 2012-2014 by abnTeX2 group at http://abntex2.googlecode.com/
%%
%% This work may be distributed and/or modified under the
%% conditions of the LaTeX Project Public License, either version 1.3
%% of this license or (at your option) any later version.
%% The latest version of this license is in
%%   http://www.latex-project.org/lppl.txt
%% and version 1.3 or later is part of all distributions of LaTeX
%% version 2005/12/01 or later.
%%
%% This work has the LPPL maintenance status `maintained'.
%%
%% The Current Maintainer of this work is the Evandro Coan.
%%
%% The last Maintainer of this work was the abnTeX2 team, led
%% by Lauro César Araujo. Further information are available on
%% https://www.abntex.net.br/
%%
%% This work consists of a bunch of files. But originally there were 2 files
%% which are renamed as follows:
%% Deleted the `abntex2-modelo-img-marca.pdf`
%% Renamed the `abntex2-modelo-include-comandos.tex, v-1.9.2 laurocesar` to `chapters/chapter_1.tex`
%%
% ---
% Este capítulo, utilizado por diferentes exemplos do abnTeX2, ilustra o uso de
% comandos do abnTeX2 e de LaTeX.
% ---

% The \phantomsection command is needed to create a link to a place in the document that is not a
% figure, equation, table, section, subsection, chapter, etc.
% https://tex.stackexchange.com/questions/44088/when-do-i-need-to-invoke-phantomsection
\phantomsection



\chapter{Revisão bibliográfica}%\label{cap:Revisão bibliográfica}

Este capítulo apresenta as principais publicações já realizadas no tema do trabalho,
%os materiais utilizados, 
seus resultados obtidos, e possíveis oportunidades de ampliação das metodologias introduzidas e análise dos resultados.

\section{Principais estudos da área}

%% Foram realizadas pesquisas nas bases de dados Web of Science, Springer, Sciencedirect e Google Scholar, de artigos científicos em inglês publicados nos últimos 10 anos (de 2010 até 2020) utilizando como palavras-chave “Dynamic, Torque, Shaft, Sensor, Strain, Gauge”, nas 3 ultimas bases foram encontrados números muito elevados de trabalhos, de ordem de centenas, a maioria dos trabalhos observados não tinham relação direta com o assunto a ser desenvolvido, a pesquisa na base de dados Web of Science obteve 10 resultados, todos eles com considerável potencial de importância para este trabalho, dentre eles 4 artigos eram diretamente desenvolvimento de metodologias similares ao trabalho.
%% NAO EH IMPORTANTE

O primeiro artigo analisado, desenvolvido por Niedworok et al em 2014 relata o desenvolvimento e aplicação de um sistema de sensoriamento de torque em tempo real em um eixo cardan de um carro de mina utilizando a medição da deformação utilizando extensômetro com transferência dos dados via radiofrequência, o trabalho também indica que o posicionamento do sensor necessita estar em contato com a superfície de maior deformação do componente, o autor realiza uma análise por elementos finitos para encontrar esse local. O artigo também aponta que o sinal vindo do sensor deve ser ampliado utilizando uma ponte de Wheatstone para conseguir ter a instrumentação correta da grandeza. O artigo mostrou resultados satisfatórios e não discutiu sobre ruídos e imprecisões presentes nos dados obtidos.

Um trabalho feito por Nurprasetio et al, em 2018, desenvolve um sistema de medição para veículos terrestres, aplicado em uma bancada de testes que simula o estado de veículos terrestres em operação, o sistema utiliza um microprocessador Arduino nano de fácil acesso e baixo custo.
Os dados são transmitidos via comunicação bluetooth.
O artigo também ilustra o processo de calibração do dispositivo feito antes do teste dinâmico, assim como no trabalho anterior, também é enfatizada a necessidade das metodologias instrumentação do sinal vindo do extensômetro.
Seus resultados também se mostraram promissores, porem o autor indica que é necessário a remoção dos ruídos de medição, o que segundo ele será endereçado em um trabalho futuro.

Um artigo feito por Gharghan et al em 2017 compara um sistema de medição similar ao dos dois trabalhos prévios com um sistema de medição de torque em tempo real de alto custo utilizado por ciclistas profissionais no pedivela.
O artigo introduz a tecnologia de transmissão de dados Zigbee, que consegue transmitir dados a a um baixo consumo energético.
Após a obtenção dos dados, o autor utiliza as ferramentas de análise estatística de Bland-Altman e porcentagem de erro médio absoluto para a validação do sistema.

O último artigo foi publicado por Silva et al em 2019 e compara os dados de um sistema semelhante aos anteriores com resultados de análises de modelo matemático analítico e analise por elementos finitos aplicados em bancadas de viga engastada com carga na ponta e de torque aplicado em um eixo com um dos lados travados, diferente dos trabalhos anteriores, esse possui uma seção com o desenvolvimento das equações dos modelos utilizados, e assim como os artigos anteriores foram encontrados resultados satisfatórios.

**colocar imagens referentes aos trabalhos**

%APRENDER A FAZER AS CITAÇÕES

%\section{Principais dispositivos existentes}
%
%Pesquisas via internet indicam a existência de dispositivos para sensorear torque dinâmico, os principais são classificados como transdutores de torque, dispositivo acoplado a um eixo que é conectados a uma interface externa, e dinamômetros, dispositivos de maior porte utilizados para obter curvas de torque e potência em motores.
