
\addtotextpreliminarycontent{\lang{List of Symbols}{Lista de Símbolos}}

% Devam aparecer na mesma ordem de ocorrência no texto.
\begin{simbolos}

    \item[$ F_{i} $] {Forças resultantes aplicadas no corpo no eixo i}
    \item[$ M_{i} $] {Momentos aplicados no corpo no eixo i}

    \item[$ \varepsilon $] {Deformação presente no material}
    \item[$ l $] {Comprimento do corpo após deformação}
    \item[$ l_0 $] {Comprimento do corpo sem deformação}

    \item[$ \sigma $] {Tensão interna do material}
    \item[$ E $] {Módulo de elasticidade do material}

    \item[$ \delta $] {Variação de comprimento de uma barra sob tração}
    \item[$ P $] {Carga aplicada no corpo}
    \item[$ A $] {Área da seção transversal}
    \item[$ L $] {Comprimento da barra}

%    \item[$ \tau $] {Tensão torcional no material}
%    \item[$ G $] {Módulo de elasticidade torsional do material}
%    \item[$ r $] {Raio da seção transversal circular}
%    \item[$ \theta $] {Ângulo de deformação}
%
%    \item[$ \nu $] {Módulo de poisson do material}
%
%    \item[$ T $] {Torque aplicado no eixo}
%    \item[$ J $] {Momento polar de inércia da seção transversal}

    \item[$ M $] {Momentos fletor na viga}
    \item[$ I $] {Momento de inércia da seção transversal}
    \item[$ y $] {Distância entre limite e o centróide da área da seção transversal}

    \item[$ L $] {Distância entre força aplicada e a força em uma viga em flexão}
    \item[$ b $] {Largura da seção transversal da viga em flexão}
    \item[$ t $] {Espessura da seção transversal da viga em flexão}

    \item[$ k $] {Fator de sensiblidade do extensômetro}
    \item[$ R_s $] {Resistência nominal do extensômetro}
    \item[$ \Delta R $] {Variação de resistência no extensômetro causada pela deformação}

    \item[$ V_{out} $] {Tensão de saída da ponte de wheatstone}
    \item[$ V_{out} $] {Tensão de excitação da ponte de wheatstone}
    \item[$ R_i $] {Resistência nominal dos resistores da ponte de wheatstone}

    \item[$ \varepsilon_{i} $] {Valor de deformação no extensômetro i}

    \item[$ Gain(A) $] {Grau de amplificação do amplificador de sinal}
    \item[$ input $] {Sinal de entrada}
    \item[$ output $] {Sinal amplificado}

    \item[$ \overline{x} $] {Valor médio das amostras}
    \item[$ x_i $] {Valor nominal da amostra i}
    \item[$ n $] {Quantidade de amostras}

    \item[$ Dp $] {Desvio Padrão}

    \item[$ \chi^{2} $] {Valor de chi quadrado}
    \item[$ observed_i $] {Valor experimental observado da amostra i}
    \item[$ expected_i $] {Valor experimental esperado da amostra i}

    \item[$ f(D) $] {Função de calibração}
    \item[$ D $] {Valor obtido do sinal em bits}
    \item[$ fa $] {Fator de calibração a}
    \item[$ fb $] {Fator de calibração b}

    \item[$ NV_{high} $] {Valor nominal da carga maior}
    \item[$ NV_{low} $] {Valor nominal da carga menor}
    \item[$ D_{high} $] {Valor obtido do sinal pela aplicação da carga maior}
    \item[$ D_{low} $] {Valor obtido do sinal pela aplicação da carga menor}

    \item[$ VE $] {Valor de engenharia}
    \item[$ RM $] {Resistência média do extensômetro}
    \item[$ RC $] {Resistência de calibração}

\end{simbolos}
